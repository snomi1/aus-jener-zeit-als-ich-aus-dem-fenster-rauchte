\documentclass{scrbook}

\usepackage[ngerman]{babel} 
 \usepackage[utf8]{inputenc}
\usepackage[T1]{fontenc} 
 \usepackage{stdpage} 
 \usepackage{datetime}

\title{Aus jener Zeit als ich aus dem Fenster rauchte} 
 \author{Dominik
Rusovac} \date{Version: \\
 \today \\ \currenttime}

\begin{document}

\maketitle

\mainmatter

\noindent \textsc{Das geliebte Vorwort} 
 \vspace{0.3cm} \\ \noindent 
 Wie ich
dies hier niederzuschreiben beginne, neigt sich der November 2020 in
 Wien dem
Ende. Nicht grundlos gebe ich Zeit und Ort an, und doch möchte ich
 all jene,
die dieses Schriftstück aus eben jenem Grund zu lesen gedenken, dem
 Zeitpunkt
seines Entstehens und der zeitlichen Verortung der darin zum Besten
 gegebenen
Dinge, nämlich, um Entschuldigung bitten und auf sonstige
 (wirklich) bedeutsame
Literatur verweisen, die ich ad hoc freilich noch nicht
 benennen kann, die es
aber mit Sicherheit eines Tages geben wird. Ja, es ist
 in diesem Jahr, wo ich
gerade bin und andauernd gewesen bin, und ich glaube,
 auch sonst wo auf dieser
Welt, vielleicht eine Scharr von Zeitzeugen und
 Zeitzeuginnen enstanden, und
wenngleich ich mich als Teil dieser Scharr
 verstehe, habe ich mich doch aus
anderen Gründen dazu entschieden, mir so
 manchen Abend mit diesem Text zu
vertreiben; im wesentlichen aus ganz
 unwesentlichen Gründen. Dies hier ist
schließlich bestenfalls ein ganz gut
 gelungener Bericht mit einem Hang zu
Fantasien und Munkeleien, oder ein
 schlichter Nachtrag meiner Gedanken aus
einer Zeit in der ich allermeistens
 aus dem Fenster rauchte.


\part*{Bericht} 



\chapter*{Warum ich aus dem Fenster rauchte}
 \newpage
 \noindent Ich sitze am
Fenster eines Kabinetts. Ich denke das Folgende: es
 ist eine mir allmählich
auffallende und im gleichen Maße zunehmend
 unangenehmer werdende Dummheit, dass
ich Texte mit Vorworten zu schreiben
 beginne und als solche, schlichte Vorworte
zu sein, für zufriedenstellend
 befinde. Ich wehre mich also gegen diesen Unsinn
und sitze, wie Mensch eben
 so am Fenster dieses einem waagrechten Schacht
ähnelnden Verschlags sitzt, am
 Fenster dieses einem waagrechten Schacht
ähnelnden Verschlags, aus dem heraus
 ich dieser Tage den Passanten die
Zigarettenstummel um die Ohren fliegen
 lasse. Ich brauche bei aller
Ungeheuerlichkeit wohl kaum zu erwähnen, dass es
 mich dabei allermeistens nicht
kümmert, wo sie landen. Einmal aus dem Fenster
 geflogen, sehe ich einen Stummel
nur noch ein paar anmutige Pirouetten
 schlagen, bis er schließlich unkenntlich
geworden vermutlich irgendwann
 irgendwo auf irgendeinem Objekt landet. So, die
Vermutung. Ich muss mich ja
 selbst über diese verächtliche Angewohnheit
wundern, wo ich mir doch sonst,
 wie unter einem Zwang von Anstand, meine fertig
gerauchten Zigaretten (nicht
 selten auch jemand anderes Zigarettenstummel) in
die Hosentaschen stecke, bis
 sich mir eine moralisch vertretbare Gelegenheit
zur Entsorgung bietet. Und
 als hätte ich mich im letztem Satz nicht schon
hinreichend in
 Schadensbegrenzung geübt: in der Annahme, ein Stummel werde auf
einem
 Menschen landen, werfe ich ihn manchmal erst, wenn die mir sympathisch
erscheinenden Köpfe aus meinem Blickfeld verschwunden sind, so, die
 vermutlich
dümmste Annahme der letzten vier Sätze, könne er ja, wenn, dann
 nur auf den
unsympathischen landen.

Ich sollte diesen Bericht vielleicht damit beginnen, der Leserschaft
verständlich zu machen, warum ich zu jener Zeit wie versessen (aus dem
 Fenster)
rauchte, oder es zumindest versuchen, und so manche Umstände
 schildern, die
eben jenen hauptsächlichen Umstand meines Rauchens am Fenster
 so interessant
machen.

Nun, wie ich diesen dritten Absatz unter den misslungenen zweiten setze, sind
ungefähr vier Monate vergangen. Der Umstand, dass ich letzteres erwähnt habe
und folgendes erwähnen werde, macht mir diesen Bericht zugegebenermaßen etwas
madig, und ich fürchte, er könne mir zu blöd werden. Wie dem auch sei. Ich
befinde mich nunmehr an einem anderen Ort, in einer anderen Stadt, wo ich das
Rauchen aus dem Fenster aufgegeben habe. Es braucht einen für's erste nicht
 zu
kümmern, es sollte nur an dieser Stelle bereits erwähnt worden sein, für
 den
Fall, ich verspürte das Gefühl, ich müsse diesen Umstand sonst wo noch
 einmal
erwähnen, und könne es mir dann doch glücklicherweise verbeissen; tut
 dies doch
in Wahrheit nichts zur Sache. 

Nun gut, ich rauchte also ungemein viel. Ich habe über mögliche Ursachen für
meine anhaltende Nikotinsucht in anderen (weitestgehend imaginären) Texten
bereits ausgiebig geschrieben, werde aber womöglich auch hier (wenn die Zeit
reif ist) ein paar mehr Gedanken als den folgenden darüber verlieren: es,
dieses Nikotin, ist ein Nervengift, eine fürchterliche Droge, ein Narkotikum,
vielleicht, das sich zuzuführen, nicht weniger süchtig macht, als die
 Substanz
selbst. So gesehen, so kommt es mir vor, bin ich vielleicht eher
 süchtig nach
dem Rauchen selbst, als nach der Substanz, die ich rauche.
 Schließlich würde
ich sehr vieles rauchen, solange es sich rauchen ließe,
 ohne an seiner Wirkung
die Vorliebe für das Rauchen zu verlieren. Was sich
 anhaltend rauchen lässt,
macht (mich) süchtig. Ich würde so weit gehen, zu
 sagen, zu rauchen ist mir
unter den unbrauchbaren Beschäftigungen die
 liebste. Ich weiß von zumindest
zwei Menschen in meiner Familie, die der
 Nikotinsucht und dem Rauchen an sich
in ähnlich bemerkenswertem Maße
 verfallen sind wie ich. Nein, ich glaube nicht,
dass diese unbrauchbare \&
 unsinnige Beschäftigung für sich ein undankbares
Erbe ist, der Hang zu
 unsinnigen Beschäftigungen an sich aber vielleicht
schon.

Ich musste also naturgemäß rauchen, zu jener Zeit als ich aus dem Fenster
rauchte. Ich rauchte aus eben jenem Fenster in meinem Verschlag, da es -
 hier,
lüge ich nicht - in jenem Haus das einzige Fenster gewesen ist, aus dem
 ich
ohne weiteres rauchen konnte. Ich wohnte mit einem Idioten in einer
 Wohnung und
einer auf den Gestank, den es ihr vermutlich gerade einmal bis
 ins Vorzimmer
getrieben hätte, hätte ich tastsächlich einmal am Gang
 geraucht, empfindlichen
alten Dame, die ich dennoch leiden konnte, in einem
 Stockwerk. Der Idiot, so
möchte ich ihn nunmehr auch taufen und fortan
 nennen, ist zu jener Zeit durch
und durch idiotisch gewesen; ein Mensch, so
 stur und widersacherisch und
besserwisserisch und scheinheilig und
 unverschämt und gierig und einfach
widerwärtig zugleich, dass einem im
 Austausch mit eben jenem Idioten aus dieser
Zeit die Entrüstung aus allen
 Poren quillt. Ich muss im letzten Satz wie ein
trotziger Süchtler klingen,
 aber glauben Sie mir, wenn ich Ihnen sage, dass ich
den Idioten von Grund auf
 nicht leiden konnte? Ich sehe es als ausgeprochen
tugendhaft an, sich in
 einer Darstellung oder Diskussion gegen die Trägheit des
eigenen Erlebens zu
 wehren, die einen mit aller Kraft dazu zwingen möchte, sich
in die eigene
 Doktrine zu verschanzen, und darin mit allen Mitteln der Kunst,
Recht zu
 behalten, zu überleben zu versuchen. Ich gestehe ihm demnach zu, dass
der
 wahre Idiot in dieser leidigen Zeit vielleicht ich selbst gewesen bin, und
vielleicht bin ich noch heute einer, und dennoch, sollte nicht unerwähnt
bleiben, dass er dies vermutlich nie gesagt hätte. Sehen Sie, was ich meine?
Lassen sie mich es diplomatisch ausdrücken, ich möchte glauben, er kann genau
so wenig dafür, dass er mich nicht leiden konnte, und vielleicht nicht im
entferntesten verstanden hat, wie ich etwas dafür kann, dass ich ihn nicht
leiden konnte, und nicht im entferntesten verstanden habe. Der Idiot war mir
derart zuwider, dass mich allein seine geräuschlose Gegenwart schon zutiefst
gestört hat, und ungute Gelüste in mir aufgekommen sind. Eine Zeit lang
 konnte
und wollte ich eigentlich nicht schreiben, aber ich habe noch die
 folgende
irrsinnige Notiz aus dem Mai 2020: Es ist nicht wenig, das ich
 zutiefst
verabscheue, was das heißen mag, dass mir nicht selten selbst die
 erträglichen
Umstände zu unerträglichen werden. Es passiert mir jedoch für
 meinen Geschmack
- und bei allem Mut zum Unmut - in letzter Zeit etwas zu
 oft. [..] Es ist mir
schon lange nicht mehr so leicht gefallen, \glqq die
 Feder zu schwingen \grqq,
aber ich flüstere Leser und Leserin leise ins Ohr,
 dass ich dies hier ernst
meine, und mir keinesfalls allein um des Schreiben
 willens derart unbeliebte
Neigungen andichten würde. Denn mir ist bewusst,
 dass es mich für die meisten
Gutmenschen unzuträglich macht. Ich habe mir
 also vorgenommen, nur so
ausführlich und böse zugleich zu sein, wie es nötig
 ist, aber ich kann
pfeiffende Menschen nicht ertragen. Sein Gepfeiffe, es hat
 mich an den Rande
des Wahnsinns getrieben. Ich habe zwangs Implosionsgefahr
 mit mir selbst zu
schimpfen begonnen. Wie ein Geisteskranker also versuchte
 ich neben dem Idioten
zu existieren. Es soll als kleiner Nachtrag nicht
 unerwähnt bleiben: Ich habe
mich selbst schon öfters beim Pfeiffen erwischt.
 Um mich für's erste nicht
weiter auf diese Weise mit dem Idioten aufzuhalten:
 mir scheint, es gibt
Lebewesen, die einander im güngstigen Fall aus dem Weg
 gehen. Es steht ja
nichts dafür Hund und Katz an einen Tisch zu setzen. Was
 und wo genau der Idiot
heute ist - ich vermute, ein Idiot irgendwo in Wien -,
 das weiß ich nicht, ich
wünsche ihm jedenfalls das Beste, das habe ich immer,
 so gut ich konnte, getan.

Ich war damals, im Jänner, oder wie es andernorts heißt - das heißt in Wahrheit
übrigens, unter anderem, nunmehr hier - im \emph{Januar}, 2020
 verlassen
worden und verließ daraufhin drei Tage später die Wohnung, die ich
 mit jener
Frau, die mich endlich verlassen hatte, eine Zeit lang bewohnt
 hatte. Aus
diesem Grund suchte ich mir eine neue Bleibe. Ich sage
 \textit{endlich}, weil
wir einander gegenseitig einen Gefallen getan haben,
 sie mir damit, dass sie
mich endlich verließ, und ich ihr damit, dass ich
 mich endlich verlassen ließ.
(Ich enthalte Ihnen an dieser Stelle ein paar
 nicht ganz unwesentliche Details
vor) So bin ich schließlich im Verschlag,
 neben dem Idioten, gelandet.
Freilich
 habe ich mich in dieser Angelegenheit,
 dem, zwangs der Trennung,
Aufdembodenzerstört- und Geistlossein,
 nämlich, gut gemacht. Ich litt. Der
Idiot (übrigens) auch. Wir tranken, jeder für sich,
 manchmal aufgrund einer
kurzen Begegnung im Vorzimmer sogar gemeinsam,
 und flanierten herum.  Er war
ebenso verlassen
 worden, im Gegensatz zu mir,
 bereits sieben Monate zuvor. Ich
bin der
 Überzeugung, dass Liebeskranke zum
 Zwecke des hemmungslosen
Liebeskrankseins 
 und einem baldestmöglichen Abkommen
 von eben letzterem
Zustand, an einern Ort 
 zusammengesetzt werden sollten. So
 oder so. Ob wir
einander nun (anfangs) nur deshalb etragen konnten, weil wir
 beide Liebeskummer
hatten, oder eben nicht, spielt keine Rolle, wir
 verstanden
 einander (noch)
irgendwie und sprachen miteinander. 




%blöd, (ich gebe es ungern zu, aber) sogar ungemein blöd, angestellt; wie ein
%geistloser, als Tier, wie Sie auch, eigentlich zum Menschsein bestimmt und
%demnach (ohne Geist) wie ein Häufchen Elend, 


\end{document}
